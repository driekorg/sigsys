\documentclass{article}
 
\usepackage{color}
\usepackage{listings}
\usepackage{graphicx}
\definecolor{MyYellow}{rgb}{1,1,0.8}
 
\lstset{language=Matlab,backgroundcolor=\color{MyYellow},basicstyle=\footnotesize,numberstyle=\footnotesize,numbers=left,stepnumber=1,numbersep=5pt,breaklines=true,frame=lines,tabsize=2}
 
\author{Ruurd Moelker \and Jan Paul Posma}
\date{\today}
\title{Signalen \& Systemen \\Practicum 2}

\begin{document}
\maketitle
 
\section{Opgave 1}
In dit practicum worden geluiden geproduceerd met behulp van frequentiemodulatie volgens de functie:
$$x(t) = A(t)cos(2 \pi f_c t + I(t)cos(2 \pi f_m t + \phi_m) + \phi_c)$$

Hierin zijn: $A(t) = A_0~exp(\frac{-t}{\tau})$ en $I(t) = I_0~exp(\frac{-t}{\tau})$. De functie bellenv, hieronder gegeven, geeft de afname van $A_0$ en $I_O$.

\begin{lstlisting}
FUNCTIE BELLENV!!!!!!!!!!!!!!
\end{lstlisting}

\section{Opgave 2}

Hieronder staat de functie playfm. Deze functie neemt als invoer:
$f_c = 110$ 
$f_m = 220$ 
$\phi_c = 0$ 
$\phi_m = 0$ 
$A_0 = 10$ 
$I_0 = 10$ 
$\tau = 2$ 
$T_{dur} = 6$
$F_s = 11025$

\begin{lstlisting}
FUNCTIE PLAYFM!!!!!!!!!!!!!!
\end{lstlisting}

TODO wat gebeurt I0 veranderen. 


\section{Opgave 3}
\subsection{case 1}
a
b
Een hogere $I_0$ zorgt ervoor dat frequentie minder snel dempen. Hierdoor klinkt het geluid scherper. Bij een lage $I_0$ zijn dezelfde frequenties actief als bij een hoge $I_0$, maar deze sterven zo snel uit dat het geluid dof klinkt.

c
d
\subsection{case 4}
a
b
De waarde van $I_0$ heeft een gelijke invloed als bij de eerste case. Een hoge $I_0$ geeft een scherp geluid, een lage $I_0$ een dof geluid.
c
d


\end{document}